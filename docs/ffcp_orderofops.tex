\documentclass{article}\usepackage[]{graphicx}\usepackage[]{color}
% maxwidth is the original width if it is less than linewidth
% otherwise use linewidth (to make sure the graphics do not exceed the margin)
\makeatletter
\def\maxwidth{ %
  \ifdim\Gin@nat@width>\linewidth
    \linewidth
  \else
    \Gin@nat@width
  \fi
}
\makeatother

\definecolor{fgcolor}{rgb}{0.345, 0.345, 0.345}
\newcommand{\hlnum}[1]{\textcolor[rgb]{0.686,0.059,0.569}{#1}}%
\newcommand{\hlstr}[1]{\textcolor[rgb]{0.192,0.494,0.8}{#1}}%
\newcommand{\hlcom}[1]{\textcolor[rgb]{0.678,0.584,0.686}{\textit{#1}}}%
\newcommand{\hlopt}[1]{\textcolor[rgb]{0,0,0}{#1}}%
\newcommand{\hlstd}[1]{\textcolor[rgb]{0.345,0.345,0.345}{#1}}%
\newcommand{\hlkwa}[1]{\textcolor[rgb]{0.161,0.373,0.58}{\textbf{#1}}}%
\newcommand{\hlkwb}[1]{\textcolor[rgb]{0.69,0.353,0.396}{#1}}%
\newcommand{\hlkwc}[1]{\textcolor[rgb]{0.333,0.667,0.333}{#1}}%
\newcommand{\hlkwd}[1]{\textcolor[rgb]{0.737,0.353,0.396}{\textbf{#1}}}%
\let\hlipl\hlkwb

\usepackage{framed}
\makeatletter
\newenvironment{kframe}{%
 \def\at@end@of@kframe{}%
 \ifinner\ifhmode%
  \def\at@end@of@kframe{\end{minipage}}%
  \begin{minipage}{\columnwidth}%
 \fi\fi%
 \def\FrameCommand##1{\hskip\@totalleftmargin \hskip-\fboxsep
 \colorbox{shadecolor}{##1}\hskip-\fboxsep
     % There is no \\@totalrightmargin, so:
     \hskip-\linewidth \hskip-\@totalleftmargin \hskip\columnwidth}%
 \MakeFramed {\advance\hsize-\width
   \@totalleftmargin\z@ \linewidth\hsize
   \@setminipage}}%
 {\par\unskip\endMakeFramed%
 \at@end@of@kframe}
\makeatother

\definecolor{shadecolor}{rgb}{.97, .97, .97}
\definecolor{messagecolor}{rgb}{0, 0, 0}
\definecolor{warningcolor}{rgb}{1, 0, 1}
\definecolor{errorcolor}{rgb}{1, 0, 0}
\newenvironment{knitrout}{}{} % an empty environment to be redefined in TeX

\usepackage{alltt}
\usepackage{Sweave}
\usepackage{float}
\usepackage{graphicx}
\usepackage{hyperref}
\hypersetup{
    colorlinks=true,
    linkcolor=blue,
    filecolor=magenta,      
    urlcolor=cyan,
}

\urlstyle{same}
\usepackage{tabularx}
\usepackage{siunitx}
\usepackage{amssymb} % for math symbols
\usepackage{amsmath} % for aligning equations
\usepackage{textcomp}
\usepackage{mdframed}
\usepackage{natbib}
\bibliographystyle{..//references/styles/besjournals.bst}
\usepackage[small]{caption}
\setlength{\captionmargin}{30pt}
\setlength{\abovecaptionskip}{0pt}
\setlength{\belowcaptionskip}{10pt}
\topmargin -1.5cm        
\oddsidemargin -0.04cm   
\evensidemargin -0.04cm
\textwidth 16.59cm
\textheight 21.94cm 
%\pagestyle{empty} %comment if want page numbers
\parskip 7.2pt
\renewcommand{\baselinestretch}{1.5}
\parindent 0pt
%\usepackage{lineno}
%\linenumbers

\newmdenv[
  topline=true,
  bottomline=true,
  skipabove=\topsep,
  skipbelow=\topsep
]{siderules}
\IfFileExists{upquote.sty}{\usepackage{upquote}}{}
\begin{document}

\noindent \textbf{\Large{Notes on FFCP deliverables}}


Notes from Laura:
\begin{enumerate}
\item The landowner document should explain what the Family Forest Carbon Program is, some of the basics of why forests and forest carbon are important, and what practices are part of the FFCP. This is the one where I could really use your help.
\item The practitioner handbook, the more technical document, should very closely track the one that the Pennsylvania FFCP pilot has produced.
  \begin{enumerate}
  \item I more envisioned your help cutting and pasting the forest practices we have into this format. 
  \item This should not be something you spend very much time on because I think we mostly have what we need from the PA pilot. 
  \item And this one should stop as soon as we're at the well-organized outline, with most of the needed text cut and pasted in stage (no meeting with the PA team or graphic designer).
  \end{enumerate}
\end{enumerate}

Order of operations:
\begin{enumerate}
\item Compile the raw materials from landowner brochures, websites and other materials:
  \begin{enumerate}
  \item The Pennsylvania pilot FFCP, start with the website here: 
    \begin{enumerate}
    \item \url{https://www.forestfoundation.org/family-forest-carbon-program}
    \end{enumerate}
  \item Forester/harvester (practitioner) handbook from the Pennsylvania pilot (attached)
  \item 2 practice descriptions from Pennsylvania (I need to get you the most up to date ones) 
  \item List of 10 practices from New England FFCP (at the end of the ``outline ffcp landowner'' document) 
  \item six practice descriptions (five are attached, and removal of competing vegetation is still being edited)
  \item Outline and starting text for landowner brochure for New England FFCP (attached, ``outline ffcp landowner...'')
  \end{enumerate}
\item Organize the two documents: create outline for a landowner document and a practitioner document, and lay out when the needed material is available from Pennsylvania versus needs to be written
\item Work with Laura to divide up sections of the outline and write rough drafts of the ones where most of the content is already available (Laura will write those when we need mostly new material)
\item Go back and forth with several rounds of edits, with Laura's guidance
\item Meet once with Pennsylvania team to discuss alignment in messaging and make needed edits so that these materials are consistent with the FFCP in Pennsylvania
\item Get text into a format where it is ready for a graphic designer. Mock up suggested graphics or data tables/etc. so that designer knows what text and images need to be included
\end{enumerate}

Order of reading material:
\begin{enumerate}
\item Read the grant proposal on Trello
\item Read my ``outline ffcp landowne...'' file
\item Look at the Family Forest Carbon Program website, especially clicking through to the parts for landowners
Pause and absorb. You might now be at the optimal amount of knowledge to start writing up an outline, because you'll have seen what the grant requires, how AFF did this in PA, and my thinking on messaging. (But you won't know too much to be wonky and dangerous!)
\item When you need to know what practices we are including in our FFCP, read:
  \begin{enumerate}
  \item the end of the ``outline FFCP landowner document''
  \item the five practices manuals
  \item Laura also has a lot of additional text on each of these, but some of it is inaccurate. 
  \item If I want more content to grab and cut and paste, look on Trello under the ``Deliverables'' list, at ``Initial practices list and backgrounders''.
  \end{enumerate}
\item By this point, we should definitely be checking in about who will do what in terms of additional outlining/organization/writing. 
\item I suspect as you send me drafts of an outline, I'll be able to send you additional resources for cut-and-paste content.
\item Any time you want a change of pace, pull up the forester manual (attached) and cut and paste in relevant parts of the practices manuals from our New England FFCP.
\end{enumerate}

  
  






\end{document}
